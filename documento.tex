%%%%%%%%%%%%%%%%%%%%%%%%%%%%%%%%%%%%%%%%%%%%%%%%%%%%%%%%%%%%%%%%%%%%%%%%
%% Customizações do abnTeX2 (http://abnTeX2.googlecode.com)           %%
%% para a Universidade Federal do Ceara - UFC                         %%
%% baseado no ueceTex2                                                %%
%% This work may be distributed and/or modified under the             %%
%% conditions of the LaTeX Project Public License, either version 1.3 %%
%% of this license or (at your option) any later version.             %%
%% The latest version of this license is in                           %%
%%   http://www.latex-project.org/lppl.txt                            %%
%% and version 1.3 or later is part of all distributions of LaTeX     %%
%% version 2005/12/01 or later.                                       %%
%%                                                                    %%
%% This work has the LPPL maintenance status `maintained'.            %%
%%                                                                    %%
%% The Current Maintainer of this work is:                            %%
%%  Alexsandro Oliveira Alexandrino                                   %%
%%                                                                    %%
%% Project available on:                                              %%
%% https://github.com/sandrooliveira1501/template-tcc2                %%
%%                                                                    %%
%% Further information about abnTeX2                                  %%
%% are available on http://abntex2.googlecode.com/                    %%
%%                                                                    %%
%%%%%%%%%%%%%%%%%%%%%%%%%%%%%%%%%%%%%%%%%%%%%%%%%%%%%%%%%%%%%%%%%%%%%%%%

\documentclass[
    a4paper,          % Tamanho da folha A4
    %sumario=tradicional, %outra opção de sumário
    12pt,             % Tamanho da fonte 12pt
    chapter=TITLE,    % Todos os capitulos devem ter caixa alta
    section=title,    % Todas as secoes devem ter caixa normal
    oneside,          % Usada para impressao em apenas uma face do papel
    english,          % Hifenizacoes em ingles
    spanish,          % Hifenizacoes em espanhol
    brazil            % Ultimo idioma eh o idioma padrao do documento
]{abntex2}

\input{lib/preambulo}

%%%%%%%%%%%%%%%%%%%%%%%%%%%%%%%%%%%%%%%%%%%%%%%%%%%%%
%%          Configuracoes do ufcTex2               %%
%%%%%%%%%%%%%%%%%%%%%%%%%%%%%%%%%%%%%%%%%%%%%%%%%%%%%

% Opcoes disponiveis

\trabalhoacademico{tccgraduacao}

% Define se o trabalho eh uma qualificacao
% Coloque 'nao' para versao final do trabalho

\ehqualificacao{nao}

% Remove as bordas vermelhas e verdes do PDF gerado
% Coloque 'sim' pare remover

\removerbordasdohyperlink{sim}

% Adiciona a cor Azul a todos os hyperlinks

\cordohyperlink{nao}

%%%%%%%%%%%%%%%%%%%%%%%%%%%%%%%%%%%%%%%%%%%%%%%%%%%%%
%%          Informação sobre a IES                 %%
%%%%%%%%%%%%%%%%%%%%%%%%%%%%%%%%%%%%%%%%%%%%%%%%%%%%%

\ies{Universidade Federal do Ceará}
\iessigla{UFC}
\centro{Campus Quixadá}

%%%%%%%%%%%%%%%%%%%%%%%%%%%%%%%%%%%%%%%%%%%%%%%%%%%%%
%%        Informação para TCC de Graduacao         %%
%%%%%%%%%%%%%%%%%%%%%%%%%%%%%%%%%%%%%%%%%%%%%%%%%%%%%

%Substituir pelo seu curso
\graduacaoem{Engenharia de Software}
\habilitacao{bacharel}
\tipohabilitacao{Bacharelado}

%%%%%%%%%%%%%%%%%%%%%%%%%%%%%%%%%%%%%%%%%%%%%%
%%  Informação relacionadas ao trabalho     %%
%%%%%%%%%%%%%%%%%%%%%%%%%%%%%%%%%%%%%%%%%%%%%%

\autor{Nome Sobrenome}
\titulo{Título do Trabalho: Subtítulo}
\data{2016}
\local{Quixadá -- Ceará}

% Exemplo: \dataaprovacao{01 de Dezembro de 2016}
\dataaprovacao{}

%%%%%%%%%%%%%%%%%%%%%%%%%%%%%%%%%%%%%%%%%%%%%
%%     Informação sobre o Orientador       %%
%%%%%%%%%%%%%%%%%%%%%%%%%%%%%%%%%%%%%%%%%%%%%

\orientador{Nome do seu Orientador}
\orientadories{Universidade Federal do Ceará – UFC}
\orientadorcentro{Campus Quixadá}
\orientadorfeminino{nao} % Coloque 'sim' se for do sexo feminino

%%%%%%%%%%%%%%%%%%%%%%%%%%%%%%%%%%%%%%%%%%%%%
%%      Informação sobre o Co-orientador   %%
%%%%%%%%%%%%%%%%%%%%%%%%%%%%%%%%%%%%%%%%%%%%%

% Deixe o nome do coorientador em branco para remover do documento

\coorientador{Nome Co-orientador}
\coorientadories{Universidade Federal do Ceará - UFC}
\coorientadorcentro{Campus Quixadá}
\coorientadorfeminino{nao} % Coloque 'sim' se for do sexo feminino

%%%%%%%%%%%%%%%%%%%%%%%%%%%%%%%%%%%%%%%%%%%%%
%%      Informação sobre a banca           %%
%%%%%%%%%%%%%%%%%%%%%%%%%%%%%%%%%%%%%%%%%%%%%

% Atenção! Deixe o nome do membro da banca para remover da folha de aprovacao

% Exemplo de uso:
% \membrodabancadois{Prof. Dr. Fulano de Tal}
% \membrodabancadoisies{Universidade Federal do Ceará - UFC}

\membrodabancadois{Membro da Banca Dois}
\membrodabancadoiscentro{Campus Quixadá}
\membrodabancadoisies{Universidade Federal do Ceará - UFC}
\membrodabancatres{Membro da Banca Três}
\membrodabancatrescentro{Campus Quixadá}
\membrodabancatresies{Universidade Federal do Ceará - UFC}
%\membrodabancaquatro{Membro da Banca Quatro}
%\membrodabancaquatrocentro{Centro de Ciências e Tecnologia - CCT}
%\membrodabancaquatroies{Universidade Federal do Ceará - UFC}
%\membrodabancacinco{Membro da Banca Cinco}
%\membrodabancacincocentro{Teste}
%\membrodabancacincoies{Universidade do Membro da Banca Cinco - SIGLA}
%\membrodabancaseis{Membro da Banca Seis}
%\membrodabancaseiscentro{}
%\membrodabancaseisies{Universidade do Membro da Banca Seis - SIGLA}

%%COMENTE TUDO QUE NÃO FOR UTILIZAR

\begin{document}
    
    \captionsetup[figure]{slc=off}

    %% CORREÇÃO DE TÍTULOS
    \titleformat{\section}
    {\normalfont\normalsize\bfseries}{\thesection}{1em}{}
    \titleformat{\subsection}
    {\normalfont\normalsize\bfseries\itshape}{\thesubsection}{1em}{}
    \titleformat{\subsubsection}
    {\normalfont\normalsize\itshape}{\thesubsubsection}{1em}{}
    \titleformat{\subsubsubsection}
    {\normalfont\normalsize\itshape}{\thesubsubsubsection}{1em}{}
    %%ADICIONE MAIS CASO NECESSÁRIO


	% Elementos pré-textuais
	\imprimircapa
	\imprimirfolhaderosto{}
	%Substitua o pdf da ficha pelo seu próprio
	\imprimirfichacatalografica{elementos-pre-textuais/ficha-catalografica}
	%\imprimirerrata{elementos-pre-textuais/errata}
	\imprimirfolhadeaprovacao
	\imprimirdedicatoria{elementos-pre-textuais/dedicatoria}
	\imprimiragradecimentos{elementos-pre-textuais/agradecimentos}
	\imprimirepigrafe{elementos-pre-textuais/epigrafe}
	\imprimirresumo{elementos-pre-textuais/resumo}
	\imprimirabstract{elementos-pre-textuais/abstract}
	%Insira apenas os utilizados
	\imprimirlistadeilustracoes
	\imprimirlistadetabelas
	\imprimirlistadequadros
	%\imprimirlistadealgoritmos
	%\imprimirlistadecodigosfonte
	%\imprimirlistadeabreviaturasesiglas % funciona apenas usando o makefile
	\imprimirlistadeabreviaturasesiglasaux{elementos-pre-textuais/lista-de-abreviaturas-siglas-aux} % opcional
	\imprimirlistadesimbolos{elementos-pre-textuais/lista-de-simbolos} % opcional
	\imprimirsumario

	%Elementos textuais
	\textual
	%Sugestão de capítulos e ordem entre eles, sinta-se a vontade para alterar
	%%
%%  O texto deste template é de autoria da professora Tania Saraiva de Melo Pinheiro (Universidade Federal do Ceará - Campus Quixadá)
%%
%%
%%
%%
%%

\chapter{Introdução}

A primeira versão da introdução da monografia pode ser uma cópia da introdução do projeto de pesquisa correspondente. Esta versão inicial será revisada à medida que o texto avança. A principal revisão é quando se termina todo o trabalho. A introdução é a primeira parte a ser escrita, uma vez que guia todo o trabalho, e também a última ser revisada.
Enuncia-se o propósito geral do trabalho. A área é brevemente contextualizada, o que será aprofundado na seção trabalhos relacionados. Uma vez contextualizado percebe-se a relevância do estudo proposto bem como seu público alvo.
Ao final da introdução, ou perto do final, o objetivo geral e os específicos são enunciados na sequência do texto, sem usar marcadores. Ao longo de todo o trabalho, convém usar a palavra “objetivo” exclusivamente para se referir os objetivos do seu trabalho.
A capa e todo o texto devem ser digitados em fonte tamanho 12, e recomendamos Times New Roman. A alternativa é apenas Arial que, ao contrário do que sugere a norma, seu tamanho 11 é mais equivalente ao Times New Roman 12. Outros elementos poderão terão letra 10 pontos, como observado neste arquivo. Na dúvida, veja as normas. O texto deve ser justificado, exceto as referências no final do trabalho, que devem ser alinhadas a esquerda.
Todos os autores citados devem ter a referência incluída na lista de referências posicionada no final no trabalho.
Em trabalhos de graduação, encontram-se denominações das partes do trabalho como Capítulos ou como Seção, porque capítulo também é conhecido como seção primária. O autor pode optar pelo que preferir, desde que padronize a nomenclatura em todo o texto. Está previsto quebra de página entre as Seções primárias.
Tipicamente, a introdução é concluída apresentando cada Seção/Capítulo que a segue.
	\chapter{Trabalhos Relacionados}

No cotidiano, um bom ponto de partida para se resolver um problema é procurar soluções já existentes para utilizá-las. Costumeiramente, as soluções que já existentes não se aplicam diretamente ao nosso caso, precisando ser adaptadas. Assim, antes de se começar a resolver questões de pesquisa, é preciso conhecer o que existe de mais atual no seu tema. 

Usando a abordagem de \citeonline{wazlawick2014metodologia} para explicar a necessidade de se conhecer a área de estudo, cabe lembrar que antes de se construir uma nova ponte é importante conhecer os tipos de pontes que já existem; é preciso conhecer qual a atualidade do assunto estudado. Do contrário, pode estar construindo uma catapulta achando que se trata da melhor forma de atravessar um rio!

Para conhecer a atualidade do tema de estudo proposto, o ideal seria fazer o vasto levantamento do que se tem estudado ou praticado sobre o tema. Entretanto, em cursos de graduação em geral não há tempo para tanto, a menos que o objeto do estudo seja justamente o levantamento do estado da arte. 
Na impossibilidade de realiza-lo, deve-se pelo menos fazer um levantamento por amostragem. Tal amostra consiste de um conjunto de trabalhos relacionados: uma boa seleção de textos encontrados em periódicos e eventos relevantes para a área estudada.  O levantamento é facilitado quando se encontram materiais denominados surveys (levantamentos), podendo ser compilações de:

\begin{alineascomponto}
    \item \textbf{Estado-da-arte}: artigos que apresentem conceitos mais recentes, estabelecidos na literatura científica;
    \item \textbf{Estado-da-prática}: semelhante ao anterior, mas com foco no que está estabelecido atualmente como status quo da prática profissional.
\end{alineascomponto}

Na escrita desta seção, deve-se evitar usar a palavra “trabalho” para se referir tanto à própria pesquisa quanto à de outro autor, sugere-se:

\begin{alineascomponto}
\item em vez de “o trabalho de Bittar (2001) prevê que ...”,
\item utilizar-se simplesmente “Bittar (20010 prevê que....”.
\end{alineascomponto}

Apenas a partir do momento em que se conhece o estado da arte (ou prática), seja plenamente ou por uma amostra de trabalhos relacionados, é que o pesquisador está pronto para adequadamente identificar e possíveis pesquisas a serem realizadas. O anúncio de seus objetivos, portanto, ocorre após tecer considerações sobre o estado do conhecimento ou prática em sua área de estudo.

\section{Quando parece ser cabível o inverso}

Em algumas situações, o pesquisador tende a apresentar primeiro seu objeto de estudo e apenas depois o estado da arte. Observa-se que tais situações ocorrem quando o problema de pesquisa é extraído do cotidiano do pesquisador. Por exemplo: “não estou conseguindo bom resultado com determinado processo de trabalho, e vou pesquisar como melhorá-lo”.

Nesta situação, há uma tendência a primeiro se definir objetivo e só depois fazer um levantamento do estado da arte/prática. Mas qual seria então o propósito do tal levantamento? Buscar soluções semelhantes que auxiliem na elaboração da solução buscada? Se assim o for, então tais trabalhos relacionados não estariam também contribuindo para um refinamento na definição dos objetivos da pesquisa?
	\chapter{Fundamentação Teórica}
\label{cap:fundamentacao-teorica}

Esta Seção contém os conceitos que fundamentarão as análises. Por exemplo, se será feita a avaliação de algum artefato, aqui se deve caracterizar quais aspectos deste artefato serão considerado, ou seja, quais os critérios de análise a serem considerados no trabalho. Também se informa qual o valor desejável para cada um desses critérios.

A fundamentação teórica NÃO é um glossário de termos, em que o autor comprova que os compreende. Sua função é deixar claro qual o significado adotado para cada conceito utilizado em sua pesquisa. Por exemplo, o que é Computação em Nuvem no seu trabalho? Você abordará questões de software ou de infraestrutura, ou a visão do usuário final? Caso existam diferentes abordagens para um mesmo conceito, deixe claro qual aquela que será adotada. 

O título desta Seção pode ser alterado de “Fundamentação Teórica” para algum tema central do seu trabalho. Tipicamente, ela contém muitas citações indicando os autores que guiaram sua elaboração. Na sequência, há muitos exemplos de como se referir e citar os textos utilizados.


\section{Título de seção }


Texto texto texto texto texto texto texto texto texto texto texto texto texto texto texto texto texto texto texto texto texto texto texto texto texto texto texto texto texto texto texto texto texto texto texto texto texto texto texto texto texto texto texto texto texto.

\subsection{Título da seção terciária}

Texto texto texto texto texto texto texto texto texto texto texto texto texto texto texto texto texto texto texto texto texto texto texto texto texto texto texto texto texto texto texto texto texto texto texto texto texto texto texto texto texto texto texto texto texto.

\subsubsection{Título da seção quaternária}

Texto texto texto texto texto texto texto texto texto texto texto texto texto texto texto texto texto texto texto texto texto texto texto texto texto texto texto texto texto texto texto texto texto texto texto texto texto texto texto texto texto texto texto texto texto.
	\chapter{MATERIAIS E MÉTODOS} % ou PROCEDIMENTOS METODOLÓGICOS
\label{chap:metodologia}

Procedimentos Metodológicos relaciona-se aos passos percorridos para responder as questões de pesquisa. Deve ser detalhado o suficiente para que outro pesquisador possa reproduzir o caminho percorrido, buscando atender às características de replicabilidade e verificabilidade da ciência. Mesmo quando não é cabível se reproduzir o estudo com as mesmas pessoas, a possibilidade de replicação do método com um público diferente é característica fundamental para a evolução do conhecimento na área. 

Descrevem-se detalhadamente as etapas para a realização da pesquisa, incluindo o campo da pesquisa e a amostra de dados considerada.  Da mesma forma que uma receita culinária, ou a descrição de um algoritmo, o método científico tem uma linguagem própria a ser seguida e é preciso cumprir as tradições de cada área de pesquisa para que os resultados sejam considerados válidos.

Cada etapa da realização do trabalho deverá responder informar aspectos como: 1) objetivo da etapa, materiais utilizados, métodos de trabalho, e campo de estudo. O campo é o local onde serão coletados os dados, devendo-se informar onde, quem, e quando ocorrerá. Quando aplicável, faça referências a apêndices do seu trabalho contendo os instrumentos de coleta de dados a serem utilizados.

Cada comunidade científica possui diferentes métodos para investigar a realidade. As técnicas de pesquisa mudam conforme a natureza do estudo e de sua área, sejam elas relacionadas à coleta ou ao registro de dados.

Seguem alguns exemplos:

\begin{alineascomponto}
    	\item Experimentos, o que inclui desenvolvimento de protótipos ou produtos;
        \item Análise de documentos;
        \item Entrevistas, em duas diversas variações;
        \item Observações, em suas diversas variações.    
\end{alineascomponto}

Estas são apenas orientações gerais. É fundamental consultar o orientador quando às prática comuns para a área temática do trabalho

\section{Título da seção secundária}

Em caso de trabalhos com coleta de dados em campo, típico de áreas como IHC e gestão, por exemplo, esta seção contém a metodologia da pesquisa. Em caso de trabalhos mais voltados para desenvolvimento, ou mais ligados à matemática, esta seção será mais breve, se houver, contendo a descrição geral dos passos realizados. 

As ilustrações (fotografias, gráficos, mapas, plantas, quadros) e tabelas devem ser citados e inseridos o mais próximo possível do trecho a que se referem. O título de ilustrações  deve estar alinhado à sua esquerda; se a ilustração for pequena, pode-se adicionar uma moldura para o resultado ficar esteticamente melhor.

Figuras devem ser legíveis, ao contrário da forma como está exibida a Figura \ref{figura-1}.  

\begin{figure}[htbp]
	\centering
	\IBGEtab{
		\Caption{\label{figura-1}Organização do conhecimento/Representação do conhecimento, Organização da informação/Representação da informação}		
    }{
		\fbox{\includegraphics[scale=1.0]{figuras/figura-1}}
	}{
	\Fonte{\citeonline{smit2010temas}}
}
\end{figure}

\lipsum[1]

\begin{figure}[htbp]
	\centering
	\IBGEtab{
		\Caption{\label{figura-1}Ciclo da informação}		
    }{
		\includegraphics[scale=0.8]{figuras/figura-2}
	}{
	\Fonte{\citeonline{tristao2004sistema}}
}
\end{figure}



	\chapter{Resultados}
\label{chap:resultados}

Na Seção de resultados faz-se um relato detalhado do que foi realizado, em especial quando se tratar de desenvolvimento. Descrição sintetizada dos dados coletados, visando responder à questão que motivou a investigação. 

A qualificação da amostra é informação de procedimentos, e não da seção Dados/resultados. Por exemplo, o percentual de homens/mulheres e identificação de público por faixa etária deve estar informado nesta seção de procedimentos porque qualifica a amostra – não são dado-resultados de pesquisa.

Tabelas não devem ter bordas laterais, como exemplificado na Tabela \ref{tabela-ibge}.

\begin{table}[h!]
	\IBGEtab{
	\Caption{\label{tabela-ibge} Um Exemplo de tabela alinhada que pode ser longa ou curta, conforme padrão IBGE.}%
	}{%
		\begin{tabular}{cccccc}
			\toprule
			Nome & Nascimento & Documento &  Nascimento & Documento & Nascimento \\
			\midrule \midrule
			Maria da Silva & 11/11/1111 & 111.111.111-11 & 111.111.111-11 & 111.111.111-11 & 111.111.111-11 \\
			Maria da Silva & 11/11/1111 & 111.111.111-11 & 111.111.111-11 & 111.111.111-11 & 111.111.111-11 \\
			Maria da Silva & 11/11/1111 & 111.111.111-11 & 111.111.111-11 & 111.111.111-11 & 111.111.111-11 \\
			\bottomrule
		\end{tabular}%
	}{%
	\Fonte{Produzido pelos autores}%
	\Nota{Esta é uma nota, que diz que os dados são baseados na
		regressão linear.}%
	\Nota[Anotações]{Uma anotação adicional, seguida de várias outras.}%
}
\end{table}

Quando não há operações sobre os dados de uma tabela, utiliza-se a denominação Quadro ou simplesmente Figura. A formatação de bordas é livre para Quadros, assim como nos elementos denominados Figura (ver Quadro \label{qua:exemplo-1}).

	\begin{quadro}[h!]	
		\centering
		\UFCqua{
		    \Caption{\label{qua:exemplo-1} Exemplo de Quadro}		
		    }{
			\begin{tabular}{|c|c|c|}
				\hline
				Quisque & faucibus & pharetra \\
				\hline
				E1 & F1 & Complete coverage by a single transcript \\
				\hline
				E2 & F1 & Complete coverage by more than \\
				\hline
				E3 & F1 & Partial coverage \\
				\hline
				E4 & F1 & Partial coverage \\
				\hline
				E5 & F1 & Partial coverage \\
				\hline
				E6 & F1 & Partial coverage \\
				\hline
				E7 & F1 & Partial coverage \\
				\hline
			\end{tabular}
		}{
			\Fonte{Elaborado pelo autor}
		}
	\end{quadro}
	\chapter{Discussão}
\label{chap:discussao}

A discussão consiste em destacar os principais resultados, bem como compará-los entre si.  Nesta Seção deve-se dar sentido ao que foi encontrado, relacionando com os objetivos do estudo, e considerando o ponto de vista escolhido para análise. 

O ponto de vista da análise deve estar em conformidade com o conteúdo da Seção de Fundamentação Teórica. Repetindo: a discussão deve estar embasada nos conceitos da fundamentação teórica, ou não terá sentido ter apresentado a fundamentação teórica no mesmo texto. Mais uma vez: é preciso verificar se os critérios de análise considerados estão explicitados na seção de revisão bibliográfica/fundamentação teórica.

Em algumas monografias, poderá ser conveniente unir a Resultados e Discussão em uma única Seção. A melhor escolha será aquela que permitir a melhor apresentação do trabalho realizado e suas contribuições. Mesmo nas situações em que a separação é mais conveniente, sugere-se que comecem a ser escritas separadamente para evitar que o conteúdo excessivo de um item não iniba o detalhamento da escrita do outro item.
	\chapter{Considerações Finais}
\label{chap:consideracoes}

Parte final do texto na qual se apresentam as conclusões apoiadas no desenvolvimento do assunto. É a recapitulação sintética dos resultados obtidos. Pode apresentar recomendações e sugestões para pesquisas futuras.

Os objetivos gerais guiam todo o trabalho de redação das considerações finais. Sendo assim, pode-se começar relendo os objetivos e analisando se eles foram, ou não, comprovadamente atingidos ao longo do trabalho. 

Aqui apresentam-se os pontos positivos e negativos da jornada, sempre visando responder aos objetivos traçados na introdução da monografia.

Isaac Newton certa vez afirmou: “se vim mais longe foi por estar no ombro de gigantes”. Deve-se reconhecer, para este trabalho, a contribuição dos que o antecederam. Mostra-se de que forma autores pesquisados e trabalhos anteriores contribuíram para o desenvolvimento do seu trabalho.

É raro se encontrar citações nas considerações finais. Uma possibilidade é quando se deseja comparar os resultados obtidos com os resultados de outras pesquisas.  Por exemplo, em trabalhos de otimização de alguma coisa, e se chegou a resultados diferentes de que outro pesquisador, este outro pesquisador pode ser citado e ter seus resultados comparados com o que foi alcançado.

Responda à questão: qual a contribuição do seu trabalho? Para tanto, volte à introdução, veja a contribuição prometida para a pesquisa, e comente a contribuição agora com base nos resultados do trabalho.

Opine! Aproveite! Essa é uma rara chance em que você tem liberdade para expressar sua opinião sem precisar buscar fundamentação na bibliografia ou nos dados coletados.
Relacione os aspectos que ficaram de fora do escopo do seu trabalho, apresentando-os como sugestões de trabalhos futuros. Desta forma, outros pesquisadores poderão dar continuidade ao que você construiu

	%Elementos pós-textuais
	\begingroup
    \raggedright
	\bibliography{elementos-pos-textuais/referencias}
    \endgroup

	%\imprimirglossario %opcional
	\imprimirapendices
		% Adicione aqui os apendices do seu trabalho
		%% USE CAPSLOCK 
\apendice{LOREM IPSUM}

O que temos em nosso corpo: apêndice ou anexo? Apêndice contém o que foi desenvolvido pelo autor. Anexo contém o que foi criado por outros autores.

\label{ap:lorem-ipsum}

\lipsum[1]
		\apendice{MODELO DE CAPA}
\label{ap:modelo-de-capa}

\lipsum[1]

	\imprimiranexos
		% Adicione aqui os anexos do seu trabalho
		%% USE CAPSLOCK 
\anexo{EXEMPLO DE ANEXO}
\label{an:exemplo-de-anexo}

\lipsum[13]
		%% USE CAPSLOCK 
\anexo{DINÂMICA DAS CLASSES SOCIAIS}
\label{an:dinamica-das-classes-sociais}

\lipsum[14]
\index{AAA}
	%\imprimirindice %opcional

\end{document}
