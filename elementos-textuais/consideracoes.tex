\chapter{Considerações Finais}
\label{chap:consideracoes}

Parte final do texto na qual se apresentam as conclusões apoiadas no desenvolvimento do assunto. É a recapitulação sintética dos resultados obtidos. Pode apresentar recomendações e sugestões para pesquisas futuras.

Os objetivos gerais guiam todo o trabalho de redação das considerações finais. Sendo assim, pode-se começar relendo os objetivos e analisando se eles foram, ou não, comprovadamente atingidos ao longo do trabalho. 

Aqui apresentam-se os pontos positivos e negativos da jornada, sempre visando responder aos objetivos traçados na introdução da monografia.

Isaac Newton certa vez afirmou: “se vim mais longe foi por estar no ombro de gigantes”. Deve-se reconhecer, para este trabalho, a contribuição dos que o antecederam. Mostra-se de que forma autores pesquisados e trabalhos anteriores contribuíram para o desenvolvimento do seu trabalho.

É raro se encontrar citações nas considerações finais. Uma possibilidade é quando se deseja comparar os resultados obtidos com os resultados de outras pesquisas.  Por exemplo, em trabalhos de otimização de alguma coisa, e se chegou a resultados diferentes de que outro pesquisador, este outro pesquisador pode ser citado e ter seus resultados comparados com o que foi alcançado.

Responda à questão: qual a contribuição do seu trabalho? Para tanto, volte à introdução, veja a contribuição prometida para a pesquisa, e comente a contribuição agora com base nos resultados do trabalho.

Opine! Aproveite! Essa é uma rara chance em que você tem liberdade para expressar sua opinião sem precisar buscar fundamentação na bibliografia ou nos dados coletados.
Relacione os aspectos que ficaram de fora do escopo do seu trabalho, apresentando-os como sugestões de trabalhos futuros. Desta forma, outros pesquisadores poderão dar continuidade ao que você construiu