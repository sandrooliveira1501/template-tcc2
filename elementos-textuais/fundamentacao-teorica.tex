\chapter{Fundamentação Teórica}
\label{cap:fundamentacao-teorica}

Esta Seção contém os conceitos que fundamentarão as análises. Por exemplo, se será feita a avaliação de algum artefato, aqui se deve caracterizar quais aspectos deste artefato serão considerado, ou seja, quais os critérios de análise a serem considerados no trabalho. Também se informa qual o valor desejável para cada um desses critérios.

A fundamentação teórica NÃO é um glossário de termos, em que o autor comprova que os compreende. Sua função é deixar claro qual o significado adotado para cada conceito utilizado em sua pesquisa. Por exemplo, o que é Computação em Nuvem no seu trabalho? Você abordará questões de software ou de infraestrutura, ou a visão do usuário final? Caso existam diferentes abordagens para um mesmo conceito, deixe claro qual aquela que será adotada. 

O título desta Seção pode ser alterado de “Fundamentação Teórica” para algum tema central do seu trabalho. Tipicamente, ela contém muitas citações indicando os autores que guiaram sua elaboração. Na sequência, há muitos exemplos de como se referir e citar os textos utilizados.


\section{Título de seção }


Texto texto texto texto texto texto texto texto texto texto texto texto texto texto texto texto texto texto texto texto texto texto texto texto texto texto texto texto texto texto texto texto texto texto texto texto texto texto texto texto texto texto texto texto texto.

\subsection{Título da seção terciária}

Texto texto texto texto texto texto texto texto texto texto texto texto texto texto texto texto texto texto texto texto texto texto texto texto texto texto texto texto texto texto texto texto texto texto texto texto texto texto texto texto texto texto texto texto texto.

\subsubsection{Título da seção quaternária}

Texto texto texto texto texto texto texto texto texto texto texto texto texto texto texto texto texto texto texto texto texto texto texto texto texto texto texto texto texto texto texto texto texto texto texto texto texto texto texto texto texto texto texto texto texto.