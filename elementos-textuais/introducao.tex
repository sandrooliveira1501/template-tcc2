%%
%%  O texto deste template é de autoria da professora Tania Saraiva de Melo Pinheiro (Universidade Federal do Ceará - Campus Quixadá)
%%
%%
%%
%%
%%

\chapter{Introdução}

A primeira versão da introdução da monografia pode ser uma cópia da introdução do projeto de pesquisa correspondente. Esta versão inicial será revisada à medida que o texto avança. A principal revisão é quando se termina todo o trabalho. A introdução é a primeira parte a ser escrita, uma vez que guia todo o trabalho, e também a última ser revisada.
Enuncia-se o propósito geral do trabalho. A área é brevemente contextualizada, o que será aprofundado na seção trabalhos relacionados. Uma vez contextualizado percebe-se a relevância do estudo proposto bem como seu público alvo.
Ao final da introdução, ou perto do final, o objetivo geral e os específicos são enunciados na sequência do texto, sem usar marcadores. Ao longo de todo o trabalho, convém usar a palavra “objetivo” exclusivamente para se referir os objetivos do seu trabalho.
A capa e todo o texto devem ser digitados em fonte tamanho 12, e recomendamos Times New Roman. A alternativa é apenas Arial que, ao contrário do que sugere a norma, seu tamanho 11 é mais equivalente ao Times New Roman 12. Outros elementos poderão terão letra 10 pontos, como observado neste arquivo. Na dúvida, veja as normas. O texto deve ser justificado, exceto as referências no final do trabalho, que devem ser alinhadas a esquerda.
Todos os autores citados devem ter a referência incluída na lista de referências posicionada no final no trabalho.
Em trabalhos de graduação, encontram-se denominações das partes do trabalho como Capítulos ou como Seção, porque capítulo também é conhecido como seção primária. O autor pode optar pelo que preferir, desde que padronize a nomenclatura em todo o texto. Está previsto quebra de página entre as Seções primárias.
Tipicamente, a introdução é concluída apresentando cada Seção/Capítulo que a segue.