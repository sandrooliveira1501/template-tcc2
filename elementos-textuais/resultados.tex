\chapter{Resultados}
\label{chap:resultados}

Na Seção de resultados faz-se um relato detalhado do que foi realizado, em especial quando se tratar de desenvolvimento. Descrição sintetizada dos dados coletados, visando responder à questão que motivou a investigação. 

A qualificação da amostra é informação de procedimentos, e não da seção Dados/resultados. Por exemplo, o percentual de homens/mulheres e identificação de público por faixa etária deve estar informado nesta seção de procedimentos porque qualifica a amostra – não são dado-resultados de pesquisa.

Tabelas não devem ter bordas laterais, como exemplificado na Tabela \ref{tabela-ibge}.

\begin{table}[h!]
	\IBGEtab{
	\Caption{\label{tabela-ibge} Um Exemplo de tabela alinhada que pode ser longa ou curta, conforme padrão IBGE.}%
	}{%
		\begin{tabular}{cccccc}
			\toprule
			Nome & Nascimento & Documento &  Nascimento & Documento & Nascimento \\
			\midrule \midrule
			Maria da Silva & 11/11/1111 & 111.111.111-11 & 111.111.111-11 & 111.111.111-11 & 111.111.111-11 \\
			Maria da Silva & 11/11/1111 & 111.111.111-11 & 111.111.111-11 & 111.111.111-11 & 111.111.111-11 \\
			Maria da Silva & 11/11/1111 & 111.111.111-11 & 111.111.111-11 & 111.111.111-11 & 111.111.111-11 \\
			\bottomrule
		\end{tabular}%
	}{%
	\Fonte{Produzido pelos autores}%
	\Nota{Esta é uma nota, que diz que os dados são baseados na
		regressão linear.}%
	\Nota[Anotações]{Uma anotação adicional, seguida de várias outras.}%
}
\end{table}

Quando não há operações sobre os dados de uma tabela, utiliza-se a denominação Quadro ou simplesmente Figura. A formatação de bordas é livre para Quadros, assim como nos elementos denominados Figura (ver Quadro \label{qua:exemplo-1}).

	\begin{quadro}[h!]	
		\centering
		\UFCqua{
		    \Caption{\label{qua:exemplo-1} Exemplo de Quadro}		
		    }{
			\begin{tabular}{|c|c|c|}
				\hline
				Quisque & faucibus & pharetra \\
				\hline
				E1 & F1 & Complete coverage by a single transcript \\
				\hline
				E2 & F1 & Complete coverage by more than \\
				\hline
				E3 & F1 & Partial coverage \\
				\hline
				E4 & F1 & Partial coverage \\
				\hline
				E5 & F1 & Partial coverage \\
				\hline
				E6 & F1 & Partial coverage \\
				\hline
				E7 & F1 & Partial coverage \\
				\hline
			\end{tabular}
		}{
			\Fonte{Elaborado pelo autor}
		}
	\end{quadro}